\section{Dinamični sistemi} \label{sec:dis}

V splošnem je dinamični sistem množica stanj skupaj z determinističnim evolucijskim pravilom. Množica stanj je običajno metrični ali topološki prostor \(X\), deterministično evolucijsko pravilo pa preslikava \(F \colon X \to X\). Dinamične sisteme delimo na

\begin{enumerate}
    \item \emph{diskretne} ali \emph{rekurzivne}, kjer je \(x_0 \in X\) in \(x_{n + 1} = F (x_n)\) za \(n \in \NN \cup \{0\}\);
    \item \emph{zvezne}, ki so sistemi diferencialnih enačb: \(\dot{\mathbf{x}} = F (\mathbf{x})\) za \(\mathbf{x} \in X \subseteq \RR^n\).
\end{enumerate}

\noindent V zgornjih primerih sta indeksni ali \emph{časovni} množici \(\NN\) in \(\RR\). Splošneje je to lahko poljuben monoid (glej \cite{Giunti_2012}).

\begin{zgled}
    Kompleksna števila \(\CC\) skupaj s kompleksno eksponentno preslikavo \(f \colon \CC \to \CC\); \(z \mapsto e^{z}\) so diskretni dinamični sistem.
\end{zgled}

\noindent Za \(n \in \NN \cup \{0\}\) označimo \(f^n \coloneq f \circ f^{n - 1}\), kjer je \(f^0 = \Id\) identiteta.

\begin{definicija}
    Naj bo \(X\) množica in \(f \colon X \to X\) preslikava. \emph{Orbita} začetne točke \(x_0 \in X\) pod preslikavo \(f\) je zaporedje \(\{f^n (x_0)\}_{n \in \NN}\). Začetni točki pravimo \emph{periodična}, če obstaja tak \(n \in \NN\), da je \(f^n (x_0) = x_0\). Če je \(n\) najmanjše tako število, pravimo, da ima periodična točka periodo \(n\). Periodična točka je \emph{fiksna}, če je njena perioda \(\num{1}\).
\end{definicija}

\begin{definicija}
    Naj bo \((X, d)\) metrični prostor, \(f \colon X \to X\) zvezna in \(x^*\) fiksna točka \(f\). Če obstaja tak \(r > 0\), da na odprti krogli \(B (x^*, r)\) orbita vsake začetne točke enakomerno konvergira proti \(x^*\), potem je \(x^*\) \emph{privlačna}. Če obstaja \(r > 0\), da za vsak \(x_0 \in B (x^*, r)\) in \(x_0 \neq x^*\), obstaja \(n \in \NN\), tako da \(f^n (x_0) \notin B (x^*, r)\).
\end{definicija}

\begin{izrek}
    Naj bo \(f \colon \CC \to \CC\) cela funkcija in \(z^*\) njena fiksna točka. Potem je \(z^*\) privlačna natanko tedaj, ko je \(|f' (z^*)| < 1\).
\end{izrek}

\begin{dokaz}
    Naj bo \(z^*\) privlačna. Ker za vsak \(z \in \Delta (z^*, r)\) zaporedje \(f^n (z)\) enakomerno konvergira proti \(z^*\), obstaja \(n_0 \in \NN\), tako da \(|f^n (z) - z^*| < r\) za vsak \(z \in \Delta (z^*, r)\) in \(n \geq n_0\). Opazujemo funkcijo \(\phi (z) = (z - z^*) / r\)
\end{dokaz}

\begin{definicija}
    Naj bo \(U \subseteq \CC\) odprta in \(f \colon U \to \CC\) holomorfna. Naj bo \(w \in U\) fiksna točka preslikave \(f\). Število \(\lambda \coloneq f' (w)\) imenujemo \emph{večkratnost preslikave \(f\) v točki \(w\)}. Negibna točka \(w\) je
    \begin{itemize}
        \item \emph{superprivlačna}, če je \(\lambda = 0\);
        \item \emph{privlačna}, če je \(|\lambda| < 1\);
        \item \emph{odbojna}, če je \(|\lambda| > 1\);
        \item \emph{racionalno nevtralna}, če je \(|\lambda| = 1\) in \(\lambda^n = 1\) za kak \(n \in \NN\);
        \item \emph{iracionalno nevtralna}, če je \(|\lambda| = 1\) in \(\lambda^n \neq\) za vsak \(n \in \NN\).
    \end{itemize}
\end{definicija}

\begin{definicija}[Topološka tranzitivnost]
    Naj bo \((X, \tau)\) topološki prostor in \(f \colon X \to X\) zvezna preslikava. Pravimo, da je \(f\) \emph{topološko tranzitivna}, če za vsaki \(U, V \in \tau\), obstajata \(z \in U\) in \(n \in \NN \cup \set{0}\), da je \(f^n (z) \in V\).
\end{definicija}

\begin{trditev}
    Naj bo \((M, d)\) metrični prostor brez izoliranih točk in \(f \colon M \to M\) zvezna preslikava. Če ima \(f\) gosto orbito, to je obstaja element, katerega orbita je gosta podmnožica \(M\), je \(f\) topološko tranzitivna.
    % Če je \(M\) separabilen in \(\num{2}\)-števen, potem topološka tranzitivnost implicira gosto orbito.
\end{trditev}

\begin{dokaz}
    Naj bo \(M\) brez izoliranih točk in \(\zap{z}\) gosta orbita. Potem za neprazni odprti množici \(U, V \subseteq M\) obstajata \(x_k \in U\) in \(x_k \in V \setminus \{x_0, x_1, \dots, x_k\}\). Slednja množica je prav tako odprta in neprazna. Ker je \(m > k\) in \(f^{m - k} (U) \cap V \neq \emptyset\), je \(f\) topološko tranzitivna.

    % Naj bo \(M\) separabilen in \(\num{2}\)-števen ter naj \(f\) nima goste orbite. Naj bo \(\zap{V}\) števna baza prostora. Za vsak \(x \in M\) obstaja \(V_{n (x)}\), tako da za vsak \(k \geq 0\) velja \(f^k (x) \notin V_{n (x)}\). Ker pa je \(f\) topološko tranzitivna je
    % \[\bigcup_{k = 0}^{\infty} f^{- k} \prt{V_{n (x)}}\]
    % odprta množica
\end{dokaz}

\begin{definicija}[Devaneyjev kaos]
    Naj bo \((M, d)\) metrični prostor in \(X \subseteq M\) neskončna. Pravimo, da je zvezna preslikava \(f \colon X \to X\) \emph{kaotična} (po Devaneyju), če sta izpolnjena naslednja pogoja.
    \begin{enumerate}
        \item Množica periodičnih točk je gosta v množici \(X\).
        \item Preslikava \(f\) je topološko tranzitivna.
    \end{enumerate}
\end{definicija}

\begin{zgled}
    Na krožnici \(S^1 \subset \CC\) definiramo rotacijo za kot \(\theta\):
    \[R_\theta \colon S^1 \to S^1; \qquad R_\theta (z) = e^{i \theta \pi} z.\]
    Naj bo \(\theta_1 = 1\). Potem je vsaka točka periodična s periodo \(\num{2}\), preslikava pa ni topološko  tranzitivna. Naj bo \(\theta_2 = p\), kjer je \(p \in \RR \setminus \QQ\). Potem je preslikava topološko tranzitivna, a niti ena točka ni periodična.
\end{zgled}

\begin{zgled}[Podvojitvena preslikava] \label{ex:double}
    Naj bo \(f \colon [0, 1) \to [0, 1)\) preslikava, podana z
    \[
        f (x) =
        \begin{cases}
            2x & \text{za } 0 \leq x < \frac{1}{2}\\
            2x - 1 & \text{za } \frac{1}{2} \leq x < 1.
        \end{cases}
    \]
    Trdimo, da je \(f\) kaotična. Označimo \(f_1 (x) = 2 x\) in \(f_2 (x) = 2x - 1\)

    Naj bo \(q \in \NN\) liho število in \(A = \set{\frac{1}{q}, \frac{2}{q}, \dots, \frac{q - 1}{q}} \subset [0, 1)\). Ni težko videti, da je \(f \colon A \to A\) injektivna in ker je \(A\) končna, je tudi bijektivna. Torej je vsak ulomek z lihim imenovalcem periodična točka. Ker so takšni ulomki gosti v \([0, 1)\), ima \(f\) gosto podmnožico periodičnih točk.

    Za dokaz topološke tranzitivnosti pokažemo, da ima \(f\) gosto orbito. Pomagamo si z zapisom števila v dvojiški bazi. Opazimo, da \(f\) ``odreže'' prvo decimalko števila. Naj bo \(\alpha \in [0, 1)\) tako, da vsebuje vsa končna zaporedja:
    \[\alpha \coloneq 0, \underbrace{0 1}_{\text{ena}} \underbrace{00011011}_{\text{dva}} \underbrace{000 001 010 100 011 101 110 111}_{\text{tri}} \dots_{(2)}\]
    Naj bo \(x = 0, d_1 d_2 d_3 \dots_{(2)}\) in \(\varepsilon > 0\). Izberemo tak \(n \in \NN\), da je \(2^{- n} < \varepsilon\). Potem obstaja tak \(k \in \NN\), da se število \(f^k (\alpha)\) začne kot \(0, d_1 d_2 \dots d_n d_{n + 1}\), torej da se \(f^k (\alpha)\) in \(x\) začneta razlikovati kvečjemu po \((n + 2)\)-ti decimalki. Potem velja
    \[\abs{f^k (\alpha) - x} \leq \sum_{i = n + 2}^{\infty} \frac{2}{2^{i}} < 2^{- n} < \varepsilon.\]
\end{zgled}

\noindent Pojem kaosa je leta \num{1989} definiral ameriški matematik Robert L.~Devaney \cite{Devaney_1986}. Za kompleksno eksponentno preslikavo kot posledica sledi iz izreka  \ref{thm:orbits}. Ta izrek je prvi dokazal poljski matematik Micha\l\ Misiurewicz leta \num{1981} \cite{Misiurewicz_1981}. S tem je potrdil domnevo, ki jo je leta \num{1926} postavil francoski matematik Pierre J.~L.~Fatou \cite{Fatou_1926}.

\vspace{1cm}

\noindent \textbf{Občutljivost na začetne pogoje}.
Devaney je v svoji prvotni definiciji zahteval tudi naslednjo lastnost.

\begin{definicija}[Občutljivost na začetne pogoje]
    Naj bo \((M, d)\) metrični prostor in \(f \colon X \to X\) zvezna preslikava. Pravimo, da je \(f\) občutljiva na začetne pogoje, če obstaja \emph{občutljivostna konstanta} \(\Delta > 0\), da za vsak \(x \in M\) in vsak \(\varepsilon > 0\) obstaja \(y \in M\), da je \(d (x, y) < \varepsilon\) in \(d (f^N (x), f^N (y)) \geq \Delta\) za nek \(N \in \NN\).
\end{definicija}

\noindent To pomeni, da majhna napaka \(\varepsilon\) v začetnih pogojih \(x\) privede do znatnih sprememb pri iteraciji dinamičnega sistema. Vendar pa se izkaže, da te lastnost pri definiciji kaotične preslikave v večini primerov ni potrebno zahtevati. Velja naslednji izrek.

\begin{izrek}
    Naj bo \((M, d)\) metrični prostor in \(X \subseteq M\) neskončna. Naj bo \(f \colon X \to X\) kaotična preslikava. Potem je \(f\) občutljiva na začetne pogoje, razen v primeru, ko je \(X\) sestavljen iz ene periodične orbite.
\end{izrek}

\begin{dokaz}
    Predpostavimo, da \(X\) ni zgolj periodična orbita. Ker so periodične točke goste, obstajata vsaj dve različni periodični orbiti. Če bi imeli skupno točko, bi bili enaki. Torej sta disjunktni in zato obstajata periodični točki \(q, r\), da je \(\Delta \coloneq \min \set{d (f^n (q), f^m (r)) : n, m \in \NN} / 8 > 0\). Pokazali bomo, da je \(\Delta\) občutljivostna konstanta.

    Če je \(x \in X\), potem je orbita ene od točk \(q, r\) vedno oddaljena od \(x\) za vsaj \(4 \Delta\). V nasprotnem primeru, ko bi bile obe v neki točki oddaljene od \(x\) za manj kot \(4 \Delta\), bi bila razdalja med orbitama manjša od \(8 \Delta\). Brez škode za splošnost je to orbita \(q\).

    Naj bo \(\varepsilon \in (0, \Delta)\). Ker so periodične točke goste, obstaja periodična točka \(p \in B (x, \varepsilon)\) s periodo \(n\). Naj bo \(V\) množica točk, ki so po \(n\) iteracijah oddaljene od \(q\) za kvečjemu \(\Delta\):
    \[V \coloneq \bigcap_{i = 0}^n f^{- i} (B (f^i (q), \Delta)).\]
    Po topološki tranzitivnosti obstaja \(k \in \NN\), da je \(f^k (B (x, \varepsilon)) \cap V \neq \emptyset\). Torej obstaja \(y \in B (x, \varepsilon)\), da je \(f^k (y) \in V\). Če definiramo \(j \coloneq \lfloor k / n \rfloor + 1\), potem je \(k / n < j \leq (k / n) + 1\) in
    \[k = n \cdot \frac{k}{n} < n j \leq n \prt{\frac{k}{n} + 1} = k + n.\]
    Torej, če je \(N \coloneq n j\), potem je \(0 < N - k \leq n\). Ker je \(f^N (p) = p\), po trikotniški neenakosti dobimo
    \begin{align}
        d \prt{f^N (p), f^N (y)} &= d \prt{p, f^N (y)} \nonumber\\
        &\geq d \prt{x, f^{N - k} (q)} - d \prt{f^{N - k} (q), f^N (y)} - d (p, x) \label{eqn:ocena}\\
        &\geq 4 \Delta - \Delta - \Delta = 2 \Delta. \nonumber
    \end{align}
    Tu smo upoštevali \(p \in B (x, \varepsilon) \subset B (x, \Delta)\) in
    \[f^N (y) = f^{N - k} \prt{f^k (y)} \in f^{N - k} (V) \subset B \prt{f^{N - k} (q), \Delta},\]
    kar velja po definiciji \(V\). Vemo, da \(p, y \in B (x, \varepsilon)\) in zato po (\ref{eqn:ocena}) velja
    \[d (f^N (p), f^N (x)) \geq \Delta \qquad \text{ali} \qquad d (f^N (y), f^N (x)) \geq \Delta.\]
\end{dokaz}

\noindent Vseeno pa je občutljivost na začetne pogoje pomembna pri obravnavi splošnih dinamičnih sistemov. V javnosti je bolj znana kot \emph{metuljev učinek}. Ime izhaja iz vprašanja, ki ga je meteorolog Ed Lorenz leta \num{1972} postavil na srečanju Ameriškega združenja za napredek znanosti: ``Ali lahko zamah kril metulja v Braziliji povzroči tornado v Teksasu?'' Vprašanje služi kot dobra prispodoba za občutljivost na začetne pogoje, a ga ne smemo vzeti dobesedno, saj ne drži \cite{Pielke_2024}.

\begin{primer}
    Eksponentna funkcija \(f \colon \RR \to \RR\); \(x \mapsto e^x\) je občutljiva na začetne pogoje za običajno metriko. Ker je za vsak \(x \in \RR\) po prvi iteraciji \(e^x > 0\), se lahko omejimo na pozitivna realna števila. Naj bo \(x > 0\) in \(\varepsilon > 0\). Definiramo \(y = x + \varepsilon / 2\). Potem \(d (x, y) = \varepsilon / 2 < \varepsilon\) in
    \[e^{y} - e^{x} = e^{x + \varepsilon / 2} - e^x = e^x \prt{e^{\varepsilon / 2} - 1} > e^{\varepsilon / 2} - 1 > \frac{\varepsilon}{2}.\]
    Zaporedje razdalj med iteracijama \(x\) in \(y\) torej divergira.
\end{primer}

\noindent Iz definicije je jasno, da je občutljivost na začetne pogoje odvisna od metrike. Zanimivo je vprašanje, ali obstajajo različno metrike, torej da preslikava lastnost ima v eni metriki, a ne v drugi. Iz tega vidika je zanimiva \emph{sferična metrika}.

\begin{definicija}
    Naj bosta \(z, w \in \CC \cup \{\infty\}\) točki na Riemannovi sferi. \emph{Sferična metrika} je preslikava
    \begin{align*}
        \chi (z, w) \coloneq \frac{|z - w|}{\sqrt{(1 + |z|^2) (1 + |w^2|)}},\\
        \chi (z, \infty) = \frac{1}{\sqrt{1 + |z|^2}}, \quad \chi (\infty, \infty) = 0.
    \end{align*}
\end{definicija}

\begin{definicija}
    Naj bo \(f \colon \CC \to \CC\) zvezna preslikava. Pravimo, da je \(f\) \emph{v sferični metriki občutljiva na začetne pogoje}, če obstajata \(\delta > 0\) in \(R > 0\) z naslednjo lastnostjo. Za vsako neprazno odprto množico \(U \subset \CC\) obstajata \(z, w \in U\) in \(n \geq 0\), tako da velja \(|f^n (z)| \leq R\) in \(|f^n (z) - f^n (w)| \geq \delta\).
\end{definicija}

\begin{trditev}
    Naj bo \(f \colon \CC \times \CC \to [0, \infty)\) zvezna preslikava, občutljiva na začetne pogoje v sferični metriki. Potem je \(f\) občutljiva na začetne pogoje glede na poljubno metriko \(d \colon \CC \times \CC \to [0, \infty)\), ki je topološko ekvivalentna evklidski (porodi evklidsko topologijo).
\end{trditev}