\section{Uvod} \label{sec:intro}

Naj bo \(x_0\) poljubno realno število. Opazujemo, kaj se dogaja z zaporedjem iteracij ali \emph{orbito} števila \(x_0\) glede na preslikavo \(e^{x}\):
\[x_0 \mapsto e^{x_0} \mapsto e^{e^{x_0}} \mapsto e^{e^{e^{x_0}}} \mapsto \cdots\]
Hitro se prepričamo, da je vsako tako zaporedje divergentno. Če pa zaporedje opazujemo kot iteracijo kompleksne eksponentne preslikave pri kompleksnem številu:
\[z_n \coloneqq e^{z_{n - 1}} \qquad \text{za } n \geq 1,\]
pa se stvar zaplete. Izkaže se, da na vsaki odprti množici obstajajo števila, katerih orbite se med seboj drastično razlikujejo.

\begin{izrek}[Orbite kompleksne eksponentne preslikave] \label{thm:orbits}
    Vsaka od naslednjih množic je za preslikavo \(f \colon \CC \to \CC\); \(z \mapsto e^{z}\) gosta v kompleksni ravnini:
    \begin{enumerate}
        \item množica števil, katerih orbita divergira k \(\infty\),
        \item množica števil, katerih orbita gosto pokrije kompleksno ravnino,
        \item množica periodičnih točk, to je števil \(z_0\), za katere obstaja
            \(k > 0\), da je \(z_k = z_0\).
    \end{enumerate}
\end{izrek}

\noindent Iz izreka nemudoma sledi, da je eksponentna preslikava \emph{kaotična}: pojem, ki ga bomo natančno definirali v razdelku \ref{sec:dis}. Za dokaz bomo potrebovali nekaj kompleksne analize ter rezultatov iz hiperbolične geometrije, ki so obravnavani v razdelku \ref{sec:hipgeom}. Preostanek dela je namenjen dokazu.