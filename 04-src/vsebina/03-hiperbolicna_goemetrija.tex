\section{Hiperbolična geometrija} \label{sec:hipgeom}

\noindent Hiperbolična geometrija se od evklidske razlikuje v tem, da aksiom

\begin{aksiom}
    Za poljubno premico \(p\) in točko \(A\), ki ni na njej, obstaja natanko ena premica \(q\), ki gre skozi \(A\) in ne seka premice \(p\).
\end{aksiom}

\noindent nadomestimo z

\begin{aksiom}
    Obstajata premica \(p\) in točka \(A\), ki ni na njej, tako, da obstajata vsaj dve premici \(q\) in \(r\), ki gresta skozi \(A\) in ne sekata premice \(p\).
\end{aksiom}

\noindent Nov aksiom prinaša veliko posledic, med drugim tudi drugo metriko; \emph{hiperbolična metrika}, katere lastnosti bomo uporabili pri dokazu. Metriko najprej definiramo na enotskem disku \(\DD \coloneq \set{z \in \CC : |z| < 1}\) in jo nato razširimo na bolj splošne množice. Za začetek ponovimo nekaj osnovnih pojmov iz kompleksne analize.

\begin{definicija}
    Naj bo \(D\) odprta v \(\CC\) in \(f \colon D \to \CC\) funkcija. Če obstaja
    \[f' (a) \coloneq \lim_{z \to a} \frac{f (z) - f (a)}{z - a},\]
    tedaj število \(f' (a)\) imenujemo odvod funkcije \(f\) v točki \(a\).
    Če \(f' (a)\) obstaja za vsak \(a \in D\), potem pravimo, da je \(f\)
    \emph{holomorfna} funkcija na \(D\).
\end{definicija}

\begin{definicija}
    Neprazni odprti in povezani podmnožici kompleksnih števil pravimo \emph{območje}.
\end{definicija}

\noindent Intuitivno si \emph{enostavno povezano} območje predstavljamo kot tisto območje, ki nima ``lukenj''. Natančno pa ga definiramo definiramo z naslednjim izrekom.

\begin{izrek}[Riemannov upodobitveni izrek]
    Za vsako enostavno povezano območje \(D \subsetneq \CC\) obstaja konformni
    izomorfizem (bijektivna holomorfna funkcija) \(\phi \colon D \to \DD\).
\end{izrek}

\subsection{Hiperbolična metrika na enotskem disku}

Hiperbolično metriko podamo prek \emph{hiperboličnega ločnega elementa}:
\[\dd \rho_{\DD} (z) = \frac{2 |\dd z|}{1 - |z|^2},\]
ki nam pove, da se infinitezimalna sprememba v točki \(z\) v hiperbolični metriki izrazi kot infinitezimalna sprememba v evklidski metriki, pomnožena s t.i.~\emph{gostoto} hiperbolične metrike
\[\rho_{\DD} (z) = \frac{2}{1 - |z|^2}.\]
Iz tega izhaja naslednja definicija.

\begin{definicija}
    Naj bo \(\gamma \colon [a, b] \to \DD\) odsekoma \(C^1\) krivulja. Njena \emph{hiperbolična dolžina} je
    \[l_{\DD} (\gamma) \coloneq \int_{\gamma} \rho_{\DD} (z) \, | \dd z | = \int_{a}^{b} \frac{2 | \gamma' (t) | }{1 - | \gamma (t) |^2} \, \dd t.\]
\end{definicija}

\noindent Končno hiperbolično metriko definiramo sledeče.

\begin{definicija}
    \emph{Hiperbolična metrika} je preslikava
    \(d_{\DD} \colon \DD \times \DD \to [0, \infty)\), definirana kot
    \[d_{\DD} (z, w) \coloneq \inf_{\gamma} l_{\DD} (\gamma),\]
    kjer \(\gamma\) teče po vseh odsekoma \(C^1\) krivuljah v \(\DD\), ki
    povezujejo točki \(z\) in \(w\).
\end{definicija}

\subsection{Hiperbolična metrika na enostavno povezanih območjih}

\noindent Definicijo nam iz enotskega diska na enostavno povezana območja prenese naslednji izrek.

\begin{izrek}[Pick] \label{thm:pick}
    Naj bosta \(U, V \subsetneq \CC\) enostavno povezani območji in \(f \colon U \to V\) holomorfna preslikava. Potem na \(U\) obstaja enolično določen hiperbolični ločni element, da velja naslednje.
    \begin{enumerate}
        \item Za vsak \(z \in \DD\) velja \(\rho_{\DD} (z) = \frac{2}{1 - |z|^2}\).
        \item Preslikava \(f\) ne veča \emph{hiperboličnega odvoda}, to je \[\hder{f (z)}{U}{V} \coloneq |f' (z)| \cdot \frac{\rho_V (f (z))}{\rho_U (z)} \leq 1.\]
        \item Enakost \(\hder{f (z)}{U}{V} = 1\) velja natanko tedaj, ko je \(f\) konformni izomorfizem.
        \item Če je \(U \subsetneq V\), potem za vsak \(z \in U\) velja \(\rho_U (z) > \rho_V (z)\). 
    \end{enumerate}
\end{izrek}

\noindent Druga in tretja točka sta pripravni za eksplicitno računanje gostot hiperbolične metrike.

\begin{trditev}[Zgledi hiperbolične metrike] \label{prop:hypexamples} \mbox{}
    \begin{enumerate}
        \item Za desno polravnino \(\HH \coloneq \{ z \in \CC : \re (z) > 0 \}\)
         je \(\rho_\HH = \frac{1}{\re (z)}\).
        \item Za pas višine \(2 \pi\) je \(\rho_{\{z \in \CC : |\im (z)| < \pi\}} = \frac{1}{2 \cos (\im (z) / 2)}\).
        \item Za zarezani ravnini velja
            \[
                \rho_{\CC \setminus [0, \infty)} (z) = \frac{1}{2 |z| \sin (\arg (z) / 2)},
                \qquad
                \rho_{\CC \setminus (-\infty, 0]} (z) = \frac{1}{2 |z| \cos (\Arg (z) / 2)}.
            \]
            Tukaj je \(\arg (z) \in [0, 2 \pi)\) in \(\Arg (z) \in (- \pi, \pi]\).
    \end{enumerate}
\end{trditev}