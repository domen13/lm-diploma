\documentclass[mat1]{fmfdelo}
% \documentclass[mat1, tisk]{fmfdelo}
% Če pobrišete možnost tisk, bodo povezave obarvane,
% na začetku pa ne bo praznih strani po naslovu, …

\avtor{Lenart Miklavič}
\naslov{Kompleksna eksponentna preslikava in kaos}
\title{The Complex Exponential Map and Chaos}
\mentor{doc.~dr.~Uroš Kuzman}
\letnica{2025} 

% - povzetek v slovenščini
%   V povzetku na kratko opišite vsebinske rezultate dela. Sem ne sodi razlaga
%   organizacije dela, torej v katerem razdelku je kaj, pač pa le opis vsebine.
\povzetek{...}

% - Prevod slovenskega povzetka v angleščino.
\abstract{...}

% - klasifikacijske oznake, ločene z vejicami
%   Oznake, ki opisujejo področje dela, so dostopne na strani https://www.ams.org/msc/
\klasifikacija{..., ...}

% - ključne besede, ki nastopajo v delu, ločene s \sep
\kljucnebesede{...\sep ...}

% - angleški prevod ključnih besed
\keywords{...\sep ...}

% - angleško-slovenski slovar strokovnih izrazov
\slovar{
% \geslo{angleški izraz}{slovenski izraz}
% ...
}

\literatura{literatura.bib}
\usepackage[tracking=true,kerning=true,babel]{microtype}
\usepackage{mathtools}
\usepackage[exponent-product=\ensuremath{\cdot},output-decimal-marker={,}]{siunitx}
\usepackage{enumitem}
\usepackage{etoolbox}

\makeatletter
\usepackage{epigraph}
    \pretocmd{\@epitext}{\em}{}{}
    \apptocmd{\@epitext}{\em}{}{}
    \setlength{\epigraphrule}{0pt}
\makeatother

\newcommand{\NN}{\mathbb N}
\newcommand{\ZZ}{\mathbb Z}
\newcommand{\QQ}{\mathbb Q}
\newcommand{\RR}{\mathbb R}
\newcommand{\CC}{\mathbb C}
\newcommand{\DD}{\mathbb D}
\newcommand{\HH}{\mathbb H}

\newcommand{\TTT}{\mathcal T}
\newcommand{\OOO}{\mathcal O}
\newcommand{\DDD}{\mathcal D}

\newcommand{\dd}{\mathrm{d}}

\DeclareMathOperator{\Id}{Id}
\DeclareMathOperator{\re}{Re}
\DeclareMathOperator{\im}{Im}
\DeclareMathOperator{\Arg}{Arg}
\DeclareMathOperator{\diam}{diam}

\DeclarePairedDelimiter\absolute{\lvert}{\rvert}
\def\abs{\absolute*}
\DeclarePairedDelimiterXPP{\hderivative}[3]{}{\lVert}{\rVert}{_{#2}^{#3}}{\mathrm{D} #1}
\def\hder{\hderivative*}

\newcommand{\prt}[1]{\left( #1 \right)}
\newcommand{\set}[1]{\left\{ #1 \right\}}
\newcommand{\zap}[1]{\left\{ #1_n \right\}_{n = 1}^{\infty}}


\begin{document}

\section{Uvod} \label{sec:intro}

Naj bo \(x_0\) poljubno realno število. Opazujemo, kaj se dogaja z zaporedjem iteracij ali \emph{orbito} števila \(x_0\) glede na funkcijo \(e^{x}\):
\[x_0 \mapsto e^{x_0} \mapsto e^{e^{x_0}} \mapsto e^{e^{e^{x_0}}} \mapsto \cdots\]
Hitro se prepričamo, da je vsako tako zaporedje divergentno. Če pa zaporedje opazujemo kot iteracijo kompleksnega števila \(z_0\) glede na preslikavo \(e^z\):
\[z_0 \mapsto e^{z_0} \mapsto e^{e^{z_0}} \mapsto e^{e^{e^{z_0}}} \mapsto \cdots,\]
se stvar zaplete. Izkaže se, da na vsaki odprti množici obstajajo števila, katerih orbite se med seboj drastično razlikujejo.

\begin{izrek}[Orbite kompleksne eksponentne preslikave] \label{thm:orbits}
    Vsaka od naslednjih množic je za preslikavo \(f \colon \CC \to \CC\); \(z \mapsto e^{z}\) gosta v kompleksni ravnini:
    \begin{enumerate}
        \item množica števil, katerih orbita divergira k \(\infty\);
        \item množica števil, katerih orbita gosto pokrije kompleksno ravnino;
        \item množica periodičnih točk, to je števil \(z_0\), za katere obstaja \(k > 0\), da je \(z_k = z_0\).
    \end{enumerate}
\end{izrek}

\noindent Iz izreka nemudoma sledi, da je eksponentna preslikava \emph{kaotična}: pojem, ki ga bomo natančno definirali v razdelku \ref{sec:dis}. Za dokaz bomo potrebovali nekaj kompleksne analize ter rezultatov iz hiperbolične geometrije, ki so obravnavani v razdelku \ref{sec:hipgeom}. Preostanek dela je namenjen dokazu.




\end{document}
